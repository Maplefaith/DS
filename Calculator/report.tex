\documentclass[UTF8]{ctexart}
\usepackage{geometry, CJKutf8}
\geometry{margin=1.5cm, vmargin={0pt,1cm}}
\setlength{\topmargin}{-1cm}
\setlength{\paperheight}{29.7cm}
\setlength{\textheight}{25.3cm}

% useful packages.
\usepackage{amsfonts}
\usepackage{amsmath}
\usepackage{amssymb}
\usepackage{amsthm}
\usepackage{enumerate}
\usepackage{graphicx}
\usepackage{multicol}
\usepackage{fancyhdr}
\usepackage{layout}
\usepackage{listings}
\usepackage{float, caption}
\usepackage[dvipsnames]{xcolor}
\lstset{
    basicstyle=\ttfamily, basewidth=0.5em
}

% some common command
\newcommand{\dif}{\mathrm{d}}
\newcommand{\avg}[1]{\left\langle #1 \right\rangle}
\newcommand{\difFrac}[2]{\frac{\dif #1}{\dif #2}}
\newcommand{\pdfFrac}[2]{\frac{\partial #1}{\partial #2}}
\newcommand{\OFL}{\mathrm{OFL}}
\newcommand{\UFL}{\mathrm{UFL}}
\newcommand{\fl}{\mathrm{fl}}
\newcommand{\op}{\odot}
\newcommand{\Eabs}{E_{\mathrm{abs}}}
\newcommand{\Erel}{E_{\mathrm{rel}}}

\begin{document}

\pagestyle{fancy}
\fancyhead{}
\lhead{Maplefatih, 3220103624}
\chead{数据结构与算法项目作业}
\rhead{Dec.17th, 2024}

\section{快速使用}
进入\texttt{Calculator}文件夹,运行\texttt{make}即可编译得到可执行文件\texttt{CAL},执行\texttt{make run}可以编译并运行程序。执行\texttt{make clean}可以删除可执行文件。

\section{项目概述}

本项目旨在设计并实现一个四则运算计算器,支持加法 (+)、减法 (-)、乘法 (*)、除法 (/) 以及括号 () 运算。计算器需要能处理合法的中缀表达式,并能够识别和报告非法输入。程序采用 C++ 标准库中的容器与算法进行实现。

\textbf{本程序能够实现的功能}:
\begin{itemize}
    \item 解析并计算四则运算表达式。
    \item 支持输入有空格。
    \item 支持对负号跟随数字的情况(表示负数)进行解析。
    \item 处理有限位小数及科学计数法表达式。
\end{itemize}

\textbf{本程序不能允许的非法输入}:
\begin{itemize}
    \item 括号不匹配。
    \item 连续运算符。
    \item 除数为0。
\end{itemize}

\textbf{关于负数}:
\begin{itemize}
    \item 负号和数字之间能够有空格。如\texttt{1-- 1}是允许的。
    \item 不能够实现连续负号再加负数,如\texttt{1---1}是不被允许的。
\end{itemize}

\textbf{关于科学计数法}:
\begin{itemize}
    \item 能够支持\texttt{E}或者\texttt{e}作为科学计数法的标志。
    \item 科学计数法要求\texttt{E}或者\texttt{e}前后包含合法的数字。如\texttt{1e3}是允许的,\texttt{1E -1.2},\texttt{2.3 e1.2},\texttt{1e*1.2}是不被允许的。
    \item 指数项必须是整数。如\texttt{2.3e1.2}是不被允许的。
\end{itemize}


\section{运算器部分的设计思路}

\subsection{输入与输出}

输入:中缀表达式字符串。

输出:若输入表达式合法,输出计算结果;若输入非法,输出\texttt{ILLEGAL}。

\subsection{处理流程}

程序的主要任务是对输入的中缀表达式进行合理有效地拆解,分成操作数和操作符,然后转化为后缀表达式进行计算。具体流程如下:

\begin{enumerate}
    \item \textbf{输入解析(\texttt{partitionString}):} 将输入字符串拆分成一个个符号(操作数和操作符),并存储在一个中缀表达式列表中。在此过程中,程序会检查操作符是否合理,数字格式是否正确(支持负数、浮点数和科学计数法)等异常。
    
    \item \textbf{中缀转后缀(\texttt{infixToPostfix}):} 利用栈将中缀表达式转换为后缀表达式(逆波兰表示法),在此过程中遵循操作符优先级和括号匹配规则。左括号 \texttt{(} 入栈,右括号 \texttt{)} 弹栈直到遇到左括号,操作符按照优先级进行栈内排序。会检查括号不匹配这一异常情况。
    
    \item \textbf{计算(\texttt{calculate}):} 通过栈计算后缀表达式的值,运算过程中按操作符顺序执行加减乘除运算。对一般操作,检查操作数是否个数正确,最后运算结果是否仅为一个数等异常情况。对除法操作,检查除数是否为零这一异常情况。
    
    \item \textbf{非法输入处理:} 以上三步都有非法的检查,确保运算结果的正确性。如果发现非法情况(如除数为零、括号不匹配、连续运算符等),立即返回\texttt{false}并输出\texttt{IILEGAL}。
\end{enumerate}

\subsection{类及其功能设计}

\subsubsection{Layer 类}

\texttt{Layer} 类用于表示后缀表达式中的每个元素,可以是\textbf{操作数}(数字)也可以是\textbf{操作符}。具体包含:
\begin{itemize}
    \item \texttt{std::string operand}: 字符串表示的操作数或操作符。
    \item \texttt{int isOperator}: 指示该层是否为操作符,\texttt{-1} 表示数字,\texttt{0} 表示操作符。同样都是操作符,还用数字\texttt{2}和\texttt{1}区分高和低优先级。
    \item \texttt{double number}: 如果该层为数字,则该字段存储数字值。使用\texttt{std::stod()}函数将数字字符串转化为数字。
\end{itemize}

\subsubsection{Calculator 类}

\texttt{Calculator} 类负责整体的计算任务,包括解析输入字符串、生成后缀表达式和执行计算流程。

\begin{itemize}
    \item \textbf{构造函数}: 接受一个输入字符串并初始化计算器,自动调用\texttt{partitionString()}和\texttt{infixToPostfix()}进行输入的解析和转换。
    \item \textbf{\texttt{bool partitionString()}}: 将输入的中缀表达式字符串拆解成有效的符号。
    \item \textbf{\texttt{bool infixToPostfix()}}: 将中缀表达式转换为后缀表达式。
    \item \textbf{\texttt{bool calculate()}}: 根据后缀表达式计算结果。
\end{itemize}

\subsection{Calculator 中重要函数的实现}

本节将重点解释程序中的三个核心函数:\texttt{partitionString()}、\texttt{infixToPostfix()} 和 \texttt{calculate()}。这三个函数分别负责输入解析、表达式转换和计算执行,是程序功能实现的核心部分。

\subsubsection{partitionString() 函数}

\texttt{partitionString()} 函数的目的是对输入字符串进行解析,将数字、操作符以及科学计数法中的符号正确拆解并存入中缀表达式(\texttt{infix})容器。若输入中包含不合法的字符或格式,程序将返回 \texttt{false}。

\textbf{工作流程:}
\begin{enumerate}
    \item 初始化阶段:
    \begin{itemize}
        \item 创建空字符串 \texttt{number} 用于存储当前解析的数字。
        \item 使用布尔变量 \texttt{hasminus} 来标识当前数字是否已经包含负号。
        \item 使用布尔变量 \texttt{hasE} 来标识当前数字是否已包含科学计数法中的 "e" 或 "E"。
    \end{itemize}

    \item 遍历输入的字符串\texttt{input}:
    \begin{itemize}
        \item 对每个字符 \texttt{ch},如果是空格,则跳过该字符。(忽略空格,解析有效字符)
        \item 如果是数字或小数点,则将其添加到 \texttt{number} 字符串中。
        \item 如果遇到 "e" 或 "E"(科学计数法中的标志符),需要检查是否已经存在另一个 "e"。如果不合法,返回 \texttt{false}。
        \item 如果遇到合法运算符(包括 +、-、*、/、(、)),程序会根据当前数字的状态决定如何处理:
        \begin{itemize}
            \item 如果 \texttt{number} 为空且遇到负号,则处理为负号的转化(将 "-" 作为 "(-1)*" 处理)。
            \item 如果 \texttt{number} 非空,先将其推入中缀表达式 \texttt{infix},然后将操作符加入中缀表达式。
        \end{itemize}
        \item 如果遇到非法字符,则返回 \texttt{false}。
    \end{itemize}

    \item 处理结束:
    \begin{itemize}
        \item 如果 \texttt{number} 非空,说明最后一个数字没有被加入中缀表达式,需要将其推入。
    \end{itemize}

\end{enumerate}

\textbf{特殊情况处理}
\begin{itemize}
    \item 程序特别处理了负号,在合适的情况下将负号转化为 "-1*",若负号前面为除号,则需转化为 "-1/"以确保负号的运算符优先级正确。
    \item 对科学计数法(如 "1.23e-4")进行严格检查,确保 "e" 后的数字和符号合法。
\end{itemize}

\textbf{异常处理}
\begin{itemize}
    \item 如果遇到不合法字符或不合规的科学计数法表达格式,返回 \texttt{false}。
\end{itemize}

\textbf{设计代码如下:}

\begin{lstlisting}[ 
    language=c++,
    basicstyle=\ttfamily,
    breaklines=true,
    keywordstyle=\bfseries\color{Blue}, 
    morekeywords={}, 
    commentstyle=\itshape\color{black!50!white},
    stringstyle=\bfseries\color{PineGreen!90!black} 
    ]
bool partitionString() {
    std::string number;
    bool hasminus = false; // 标识当前number中是否已经包含e或E
    bool hasE = false; // 标识当前number中是否已经包含-号

    char ch = input[0];
    for (size_t i = 0; i < input.size(); ++i) {
        ch = input[i];
        if (std::isspace(ch)) {
            continue; // 跳过空格
        }
        if (std::isdigit(ch) || ch == '.') { // 数字或小数点直接加入number字符串
            number += ch;
        } else if (ch == 'e' || ch == 'E') { // 科学计数法中的 e 或 E 前后必须跟合法数字,且不能重用e
            if (hasE || !isvalidE(i))return false;
            hasE = true;
            number += 'e'; // 加入 e 或 E 及其之后合法元素
            ++i;
            do{
                number += input[i++];
            }
            while(std::isdigit(input[i]));
            if(input[i] == '.' || isoperatorinE(number.back()))return false; // e后出现小数点 '.' 或者 单个 '+' '-'
            i -= 1;
        } else if (isoperator(ch)) { // 合法运算符
            if(!hasminus && ch == '-' && number.empty() && (isvalidminus())) { // "-" 作为负号的转化
                std::string mark="";
                if(!infix.empty()) mark= infix.back();
                infix.push_back("-1");
                if(mark=="/")infix.push_back("/");
                else infix.push_back("*");
                hasminus=true;
                // number+=ch;
            } else {
                if(!number.empty()){
                    infix.push_back(number);
                    hasE = false;
                    number.clear();
                }
                hasminus = false;
                infix.push_back(std::string(1, ch)); // 操作符加入中缀表达式
                } 
        } else {
            return false; // 其他非法字符
        }
    }

    if (!number.empty()) infix.push_back(number);
    return true;
}
\end{lstlisting}

\subsubsection{infixToPostfix() 函数}

\texttt{infixToPostfix()} 函数的主要任务是将拆解后的中缀表达式转换为后缀表达式(即逆波兰表示法)。在中缀表达式中,操作符的位置依赖于括号和优先级,而在后缀表达式中,操作符位置则依赖于操作数顺序,且不需要括号来明确运算优先级。

\textbf{工作流程:}
\begin{itemize}
    \item 使用栈来暂存操作符和括号,操作数直接添加到后缀表达式中。
    \item 遇到左括号 \texttt{(} 时,直接入栈,表示当前运算的开始。
    \item 遇到右括号 \texttt{)} 时,弹出栈顶操作符直到遇到左括号,并将弹出的操作符加入后缀表达式。
    \item 对于其他操作符,检查栈顶的操作符是否具有更高的优先级,如果有,则弹出栈中的操作符到后缀表达式中,直到栈顶操作符优先级较低或栈为空。
    \item 最后,栈中剩余的所有操作符按顺序加入到后缀表达式中。
\end{itemize}

\textbf{异常处理:}
\begin{itemize}
    \item 在遍历中,若发现括号不匹配,函数会返回 \texttt{false}。
\end{itemize}

\textbf{代码设计如下:}
\begin{lstlisting}[ 
    language=c++,
    basicstyle=\ttfamily,
    breaklines=true,
    keywordstyle=\bfseries\color{Blue}, 
    morekeywords={}, 
    commentstyle=\itshape\color{black!50!white},
    stringstyle=\bfseries\color{PineGreen!90!black} 
    ]
bool infixToPostfix() {
    std::stack<Layer> operators;

    for (const auto& token : infix) {
        if (std::isdigit(token[0]) || (token.size() > 1 && token[0]=='-')) {
            postfix.push_back(Layer(token, -1));  // 操作数直接加入后缀表达式
        } else if (token == "(") {
            operators.push(Layer(token, 0));     // 左括号压栈
        } else if (token == ")") {
            // 遇到右括号,弹出直到左括号
            while (!operators.empty() && operators.top().getoperand() != "(") {
                postfix.push_back(operators.top());
                operators.pop();
            }
            if (operators.empty()) return false;  // 括号不匹配
            operators.pop();  // 弹出左括号
        } else {
            // 遇到运算符
            Layer temp = Layer(token, 0);
            while (!operators.empty() && operators.top().getisOperator() >= temp.getisOperator()) {
                postfix.push_back(operators.top());
                operators.pop();
            }
            operators.push(temp);
        }
    }

    // 将剩余运算符压入后缀表达式
    while (!operators.empty()) {
        if (operators.top().getoperand() == "(") return false;  // 括号不匹配
        postfix.push_back(operators.top());
        operators.pop();
    }

    return true;
}
\end{lstlisting}

\subsubsection{calculate() 函数}

\texttt{calculate()} 函数的功能是根据转换后的后缀表达式,利用堆栈结构进行运算,计算成功返回\texttt{true},将计算结果存入变量\texttt{out}中。

\textbf{工作流程:}
\begin{itemize}
    \item 遍历后缀表达式,对于每个元素:
    \begin{itemize}
        \item 如果是操作数(数字),将其压入栈中。
        \item 如果是操作符,从栈中弹出两个操作数,进行相应的运算,结果再压回栈中。
    \end{itemize}
    \item 在所有操作符都被处理完后,栈中会剩下一个元素,它就是计算结果。
    
\end{itemize}

\textbf{异常处理:}
\begin{itemize}
    \item 在进行除法运算时,需要检查除数是否为零,如果是零,函数会返回 \texttt{false}。
    \item 遇到操作数个数不正确等情况,函数会返回 \texttt{false}。
\end{itemize}

\textbf{代码设计如下:}
\begin{lstlisting}[ 
    language=c++,
    basicstyle=\ttfamily,
    breaklines=true,
    keywordstyle=\bfseries\color{Blue}, 
    morekeywords={}, 
    commentstyle=\itshape\color{black!50!white},
    stringstyle=\bfseries\color{PineGreen!90!black} 
    ]
bool calculate() {
    for (const auto& token : postfix) {
        if (token.getisOperator()==-1) {
            // 将操作数压入stack
            stack.push(token);
        } else {
            // 从stack中弹出两个操作数
            if(stack.size()<2) return false;
            Layer right = stack.top(); stack.pop();
            Layer left = stack.top(); stack.pop();

            if(right.getisOperator()!=-1 || left.getisOperator()!=-1)return false;

            double result = 0.0;
            if (token.getoperand() == "+") result = left.getnumber() + right.getnumber();
            else if (token.getoperand() == "-") result = left.getnumber() - right.getnumber();
            else if (token.getoperand() == "*") result = left.getnumber() * right.getnumber();
            else {
                if (right.getnumber() == 0) return false;  // 检查除数是否为0
                result = left.getnumber() / right.getnumber();
            }
            // 将结果作为新的操作数压回栈
            stack.push(Layer(std::to_string(result), -1));
        }
    }
    if (stack.size() != 1 || stack.top().getisOperator()!=-1) return false;
    out = stack.top().getnumber();
    return true;
}
\end{lstlisting}

\section{测试程序的设计}
\subsection{设计原则}
测试的设计基于以下几个原则和思路:

\begin{itemize}
    \item \textbf{全面性}:测试用例应涵盖各种可能的数学表达式,包括简单的运算、括号运算、负号运算、除零情况、科学计数法等。通过覆盖常见场景,确保计算器在大多数情况下都能正确计算。
    \item \textbf{错误处理}:计算器应能够识别并妥善处理非法的表达式,比如括号不匹配、运算符使用错误、除数为零等常见错误。测试中应包括这些错误类型,以验证程序的鲁棒性。
    \item \textbf{边界测试}:通过设计一些极限情况和特殊场景(如极小值、极大值、复杂的嵌套括号等),检查程序在这些情况下是否能正确计算并避免出现溢出或精度问题。
\end{itemize}

\subsection{测试内容}

测试内容分为以下几个类别:

\begin{itemize}
    \item \textbf{括号和运算符的正确性}:测试括号匹配和运算符使用的合法性,确保表达式的结构合法且符合数学规则。
    \item \textbf{除法和零的处理}:测试除数为零的情况,确保程序能够正确地捕捉并处理除以零的错误。
    \item \textbf{负号运算符的识别}:验证负号的使用是否符合规则,确保程序能够正确识别负数和负号运算。
    \item \textbf{科学计数法的解析}:验证程序对科学计数法的支持,包括正确解析正负指数以及避免误用。
    \item \textbf{连续运算符和运算顺序}:测试运算符是否能够按照正确的优先级和顺序进行运算,且不出现不合法的连续运算符。
    \item \textbf{边界情况}:包括处理非常大或非常小的数字、复杂嵌套的括号、极端运算等。
\end{itemize}


\textbf{代码设计如下:}
\begin{lstlisting}[ 
    language=c++,
    basicstyle=\ttfamily,
    breaklines=true,
    keywordstyle=\bfseries\color{Blue}, 
    morekeywords={}, 
    commentstyle=\itshape\color{black!50!white},
    stringstyle=\bfseries\color{PineGreen!90!black} 
    ]
void testCal(std::vector<std::string> &expressions){
    for (const auto& expr : expressions) {
        std::cout << expr << ": "<<std::endl;
        Calculator cal(expr);
        if(!cal.getisValid() || !cal.calculate()){
            std::cout<<"ILLIGAL"<<std::endl;
        }else {
            std::cout<< std::setprecision(8) <<cal.getresult()<<std::endl;
        }
    }
}

int main() {
    
    std::vector<std::string> exp[10];
    int end = -1;

    // 括号不匹配
    std::cout<<"----------------------------------------"<<std::endl;
    std::cout<<"括号不匹配"<<std::endl;
    exp[++end] = {  
        "(3 + 2 * (4 - 1",          // 错误:括号不匹配
        "((5 + 6) * 3) - 4)",      // 错误:括号不匹配
        "3 + (4 - 2))",            // 错误:括号不匹配
        "((2)+(2))+5+6*(2)+6)"     // 错误:括号不匹配
    };
    testCal(exp[end]);

    // 运算符连续使用
    std::cout<<"----------------------------------------"<<std::endl;
    std::cout<<"运算符连续使用"<<std::endl;
    exp[++end] = {  
        "7 --- 2",                // 错误:连续运算符
        "7 -- 2",                 // 正确: 7 - (-2) = 7 + 2 = 9
        "4 ** 2",                 // 错误:非法运算符 **
        "3 + * 2",                // 错误:运算符缺失数字
        "10 / / 5",               // 错误:连续运算符
        "(7 +-1)+- 2.5",          // 正确: (7 + (-1)) + (-2.5) = 6 - 2.5 = 3.5
        "(7 -+1)-+ 2.5",          // 错误,-+1不合法
        "(7 *-1)/- 2",            // 正确: (7 * -1) / (-2) = -7 / -2 = 3.5
        "(7 -*1)/- 2",            // 错误:运算符使用错误
    };
    testCal(exp[end]);

    // 以运算符开头或结尾
    std::cout<<"----------------------------------------"<<std::endl;
    std::cout<<"以运算符开头或结尾"<<std::endl;
    exp[++end] = { 
        "+ 5 * 3",                // 错误:以运算符开头
        "5 * 3+",                 // 错误:以运算符结尾
        "7 -",                    // 错误:以运算符结尾
        "- 7",                    // 错误:以运算符开头
        "* (4 + 5)",              // 错误:以运算符开头
        "(4 + 5) *",              // 错误:以运算符结尾
        "/ (3 + 2)",              // 错误:以运算符开头
        "(3 + 2) /"               // 错误:以运算符结尾
    };
    testCal(exp[end]);

    // 除数为 0
    std::cout<<"----------------------------------------"<<std::endl;
    std::cout<<"除数为零"<<std::endl;
    exp[++end] = { 
        "8 / 0",                  // 错误:除以零
        "10 / (3 - 3)",           // 错误:除以零
        "15 / (7 - (2 + 5))"      // 错误:除以零
    };
    testCal(exp[end]);

    // 括号和负号运算符
    std::cout<<"----------------------------------------"<<std::endl;
    std::cout<<"括号和负号运算符"<<std::endl;
    exp[++end] = {  
        "12.3*(-1)/3",            // 正确: 12.3 * (-1) / 3 = -12.3 / 3 = -4.1
        "(8.12 / -2) /1",         // 正确: 8.12 / (-2) / 1 = -4.06
        "10.10 / (2 +- 3)",       // 正确: 10.10 / (2 + (-3)) = 10.10 / (-1) = -10.10
        "-15 / (7 *-1/-(2 + 5))", // 正确: -15 / (7 * -1 / -(7)) = -15 / (7 * -1 / -7) = -15 / 1 = -15
        "-(2)-(-3)",              // 正确: -2 - (-3) = -2 + 3 = 1
        "2.3+4*((12)-(-2))",      // 正确: 2.3 + 4 * (12 + 2) = 2.3 + 56 = 58.3
        "(-(-5.2))*(-4.2)+-3/6"  // 正确: -(-5.2) = 5.2; 5.2 * (-4.2) = -21.84; -3 / 6 = -0.5; -21.84 + (-0.5) = -22.34
    };
    testCal(exp[end]);

    // 使用科学计数法
    std::cout<<"----------------------------------------"<<std::endl;
    std::cout<<"使用科学计数法"<<std::endl;
    exp[++end] = { 
        "8.12e3 / -2",            // 正确: 8.12e3 / -2 = 8120 / -2 = -4060
        "10.10e1 / (2 +- 3)",     // 正确: 10.10e1 / (2 + (-3)) = 101 / (-1) = -101
        "15e-1 / (7 *-1- (2 + 5))", // 正确: 15e-1 = 1.5; (7 * -1) - (2 + 5) = -7 - 7 = -14; 1.5 / -14 = -0.1071

        "2e3e4",                 // 错误:重复的e
        "e5 + 1",                // 错误:e不能单独出现
        "1.2e + 3",              // 错误:缺少指数
        "1 + -",                 // 错误:单独的负号
        "1e",                    // 错误:不完整的科学计数法
        "1.2e++3",               // 错误:连续符号
        "(2e-) + 3"              // 错误:括号内不完整的科学计数法
    };
    testCal(exp[end]);

    std::cout<<"----------------------------------------"<<std::endl;
    std::cout << "测试负数、括号和科学计数法的表达式" << std::endl;

    exp[++end] = {
        "1.5e2 + -(3.1E-2)",      // 正确: 1.5e2 = 150, 3.1E-2 = 0.031; 150 + (-0.031) = 149.969
        "-(2*-1.23e4) + -(5.67E+4)",    // 正确: -(2 * -1.23e4) = 24600; -(5.67e4) = -56700; 24600 + -56700 = -32100
        "3 + -1/ -2.1e+3",           // 正确: -1 / -2.1e3 = 0.000476; 3 + 0.000476 = 3.000476
        "-3.14e-2",              // 正确: -3.14e-2 = -0.0314
        "2.5e2 *1/(-1/ -3.2e1)",      // 正确: 2.5e2 * 1 / (1 / 3.2e1) = 250 * 3.2e1 = 8000
    };
    testCal(exp[end]);

    std::cout<<"----------------------------------------"<<std::endl;
    std::cout << "复杂测试" << std::endl;
    exp[++end] = {
        "((1e3 - (2e2 + 1e1)) / (3e-2 * (4e3 + -5e2)))",       // 正确: 1e3 - 2e2 - 1e1 = 790; 3e-2 * 3500 = 105; 790 / 105 = 7.523809
        "(((-5e2 + 6e3) * 3e1) / (2e4 - (3e-2 + -1e-1)))",      // 正确: (-5e2 + 6e3) = 5500; 5500 * 3e1 = 165000; 2e4 - 3e-2 + 1e-1 = 19999.97; 165000 / 19999.97 = 8.25
        "(3e2 - (2e3 + (1e4 / (-5e1 - 2e2))))",                  // 正确: 300 - (2000 + (10000 / (-50 - 200))) = -1660
    };
    testCal(exp[end]);

    return 0;
}
\end{lstlisting}
\section{测试结果}
测试结果一切正常。输出和预期相符:

\begin{lstlisting}[ 
    language=c++,
    basicstyle=\ttfamily,
    breaklines=true,
    keywordstyle=\bfseries\color{Blue}, 
    morekeywords={}, 
    commentstyle=\itshape\color{black!50!white},
    stringstyle=\bfseries\color{PineGreen!90!black} 
    ]
----------------------------------------
括号不匹配
(3 + 2 * (4 - 1: 
ILLIGAL
((5 + 6) * 3) - 4): 
ILLIGAL
3 + (4 - 2)): 
ILLIGAL
((2)+(2))+5+6*(2)+6): 
ILLIGAL
----------------------------------------
运算符连续使用
7 --- 2: 
ILLIGAL
7 -- 2: 
9
4 ** 2: 
ILLIGAL
3 + * 2: 
ILLIGAL
10 / / 5: 
ILLIGAL
(7 +-1)+- 2.5: 
3.5
(7 -+1)-+ 2.5: 
ILLIGAL
(7 *-1)/- 2: 
3.5
(7 -*1)/- 2: 
ILLIGAL
----------------------------------------
以运算符开头或结尾
+ 5 * 3: 
ILLIGAL
5 * 3+: 
ILLIGAL
7 -: 
ILLIGAL
- 7: 
-7
* (4 + 5): 
ILLIGAL
(4 + 5) *: 
ILLIGAL
/ (3 + 2): 
ILLIGAL
(3 + 2) /: 
ILLIGAL
----------------------------------------
除数为零
8 / 0: 
ILLIGAL
10 / (3 - 3): 
ILLIGAL
15 / (7 - (2 + 5)): 
ILLIGAL
----------------------------------------
括号和负号运算符
12.3*(-1)/3: 
-4.1
(8.12 / -2) /1: 
-4.06
10.10 / (2 +- 3): 
-10.1
-15 / (7 *-1/-(2 + 5)): 
-15
-(2)-(-3): 
1
2.3+4*((12)-(-2)): 
58.3
(-(-5.2))*(-4.2)+-3/6: 
-22.34
----------------------------------------
使用科学计数法
8.12e3 / -2: 
-4060
10.10e1 / (2 +- 3): 
-101
15e-1 / (7 *-1- (2 + 5)): 
-0.107143
2e3e4: 
ILLIGAL
e5 + 1: 
ILLIGAL
1.2e + 3: 
ILLIGAL
1 + -: 
ILLIGAL
1e: 
ILLIGAL
1.2e++3: 
ILLIGAL
(2e-) + 3: 
ILLIGAL
----------------------------------------
测试负数、括号和科学计数法的表达式
1.5e2 + -(3.1E-2): 
149.969
-(2*-1.23e4) + -(5.67E+4): 
-32100
3 + -1/ -2.1e+3: 
3.000476
-3.14e-2: 
-0.0314
2.5e2 *1/(-1/ -3.2e1): 
8000
----------------------------------------
复杂测试
((1e3 - (2e2 + 1e1)) / (3e-2 * (4e3 + -5e2))): 
7.52381
(((-5e2 + 6e3) * 3e1) / (2e4 - (3e-2 + -1e-1))): 
8.249971
(3e2 - (2e3 + (1e4 / (-5e1 - 2e2)))): 
-1660
\end{lstlisting}

我用 \texttt{valgrind} 进行测试,发现没有发生内存泄露。
\begin{lstlisting}[ 
    language=c++,
    basicstyle=\ttfamily,
    breaklines=true,
    keywordstyle=\bfseries\color{Blue}, 
    morekeywords={}, 
    commentstyle=\itshape\color{black!50!white},
    stringstyle=\bfseries\color{PineGreen!90!black} 
    ]
==226877== 
==226877== HEAP SUMMARY:
==226877==     in use at exit: 0 bytes in 0 blocks
==226877==   total heap usage: 615 allocs, 615 frees, 198,144 bytes allocated
==226877== 
==226877== All heap blocks were freed -- no leaks are possible
==226877== 
==226877== For lists of detected and suppressed errors, rerun with: -s
==226877== ERROR SUMMARY: 0 errors from 0 contexts (suppressed: 0 from 0)    
\end{lstlisting}



\end{document}

%%% Local Variables: 
%%% mode: latex
%%% TeX-master: t
%%% End: 
