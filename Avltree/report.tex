\documentclass[UTF8]{ctexart}
\usepackage{geometry, CJKutf8}
\geometry{margin=1.5cm, vmargin={0pt,1cm}}
\setlength{\topmargin}{-1cm}
\setlength{\paperheight}{29.7cm}
\setlength{\textheight}{25.3cm}

% useful packages.
\usepackage{amsfonts}
\usepackage{amsmath}
\usepackage{amssymb}
\usepackage{amsthm}
\usepackage{enumerate}
\usepackage{graphicx}
\usepackage{multicol}
\usepackage{fancyhdr}
\usepackage{layout}
\usepackage{listings}
\usepackage{float, caption}
\usepackage[dvipsnames]{xcolor}
\lstset{
    basicstyle=\ttfamily, basewidth=0.5em
}

% some common command
\newcommand{\dif}{\mathrm{d}}
\newcommand{\avg}[1]{\left\langle #1 \right\rangle}
\newcommand{\difFrac}[2]{\frac{\dif #1}{\dif #2}}
\newcommand{\pdfFrac}[2]{\frac{\partial #1}{\partial #2}}
\newcommand{\OFL}{\mathrm{OFL}}
\newcommand{\UFL}{\mathrm{UFL}}
\newcommand{\fl}{\mathrm{fl}}
\newcommand{\op}{\odot}
\newcommand{\Eabs}{E_{\mathrm{abs}}}
\newcommand{\Erel}{E_{\mathrm{rel}}}

\begin{document}

\pagestyle{fancy}
\fancyhead{}
\lhead{Maplefatih, 3220103624}
\chead{数据结构与算法第六次作业}
\rhead{Nov.9th, 2024}

\section{remove函数的设计思路}
设计思路和上次类似,重点在于通过指针操作完成后,对部分节点进行调整。\texttt{balance()}已经包含了高度调整和各类旋转操作,具体详见代码。要调整的节点可分为两类:
\begin{itemize}
    \item \textbf{被删除节点所在位置节点及其父节点}。这一点要求我们在递归回调时进行\texttt{balance()},故放在函数\texttt{remove()}的删除节点操作之后,调用函数结束(无返回值)之前。
    \item \textbf{右子树最小元节点所在位置节点及其父节点}。被移动后,原所在位置被替换为新节点,故也可能影响树的结构。这一点要求我们不仅利用递归来查找右子树最小元节点(不使用循环来查找),而且在函数返回前,进行树的调整。故\texttt{balance()}放在\texttt{current}节点更新完成后,调用函数结束(有返回值)之前。
\end{itemize}

具体代码参考如下。
\begin{lstlisting}[ 
    language=c++,
    basicstyle=\ttfamily,
    breaklines=true,
    keywordstyle=\bfseries\color{Blue}, 
    morekeywords={}, 
    commentstyle=\itshape\color{black!50!white},
    stringstyle=\bfseries\color{PineGreen!90!black} 
    ]
void remove(const Comparable &x, BinaryNode * &t) {
    if (t == nullptr) { return; } // 元素不存在
    if (x < t->element) { remove(x, t->left);
    } else if (x > t->element) { remove(x, t->right);
    } else if (t->left != nullptr && t->right != nullptr) {  /// 有两个子节点
        BinaryNode *current = detachMin(t->right), *oldNode = t;
        current->left = t->left; current->right = t->right;
        t = current; delete oldNode;
    } else { BinaryNode *oldNode = t; /// 有一个或没有子节点
        t = (t->left != nullptr) ? t->left : t->right; delete oldNode; 
    } balance(t);
}
BinaryNode *detachMin(BinaryNode *&t) {
    BinaryNode * current = t;
    if(t->left!=nullptr) current = detachMin(t->left);
    else{ BinaryNode *oldNode = t; t = t->right; return oldNode;
    } balance(t);
    return current;
}
\end{lstlisting}

此外,\texttt{BST.h}有其他增改。主要包括

\begin{lstlisting}[ 
    language=c++,
    basicstyle=\ttfamily,
    breaklines=true,
    keywordstyle=\bfseries\color{Blue}, 
    morekeywords={}, 
    commentstyle=\itshape\color{black!50!white},
    stringstyle=\bfseries\color{PineGreen!90!black} 
    ]
void balance( BinaryNode * & t );
int height( BinaryNode *t ) const;
int max( int lhs, int rhs ) const;
void rotateWithLeftChild( BinaryNode * & k2 );
void rotateWithRightChild( BinaryNode * & k1 );
void doubleWithLeftChild( BinaryNode * & k3 );
void doubleWithRightChild( BinaryNode * & k1 );
\end{lstlisting}
来使得删除节点后能按 AVL 树的方式保持平衡。
\end{document}

%%% Local Variables: 
%%% mode: latex
%%% TeX-master: t
%%% End: 
